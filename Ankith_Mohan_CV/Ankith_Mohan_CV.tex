%%%%%%%%%%%%%%%%%%%%%%%%%%%%%%%%%%%%%%%%%%%%%%%%%%%%%%%%%%%%%%%%%%%%
% Medium Length Professional CV
% LaTeX Template
% Version 2.0 (8/5/13)
%
% This template has been downloaded from:
% http://www.LaTeXTemplates.com
%
% Original author:
% Trey Hunner (http://www.treyhunner.com/)
%
% Important note:
% This template requires the resume.cls file to be in the same directory as the
% .tex file. The resume.cls file provides the resume style used for structuring the
% document.
%
%%%%%%%%%%%%%%%%%%%%%%%%%%%%%%%%%%%%%%%%%%%%%%%%%%%%%%%%%%%%%%%%%%%%
\documentclass{resume} % Use the custom resume.cls style

\usepackage[left=0.75in,top=0.6in,right=0.75in,bottom=0.6in]{geometry} % Document margins
\usepackage{hyperref}

\name{Ankith Mohan}
\address{\href{mailto:ankithmo@usc.edu}{ankithmo@usc.edu} \\ \href{https://ankithmo.github.io}{ankithmo.github.io} \\ }

\begin{document}
%%%%%%%%%%%%%%%%%%%%%%%%%%%%%%%%%%%%%%%%%%%%%%%%%%%%%%%%%%%%%%%%%%%%
% Education
%%%%%%%%%%%%%%%%%%%%%%%%%%%%%%%%%%%%%%%%%%%%%%%%%%%%%%%%%%%%%%%%%%%%
\begin{rSection}{Education}

{\bf MS in Computer Science} \hfill {\em 2018 - 2020} \\ 
{\it University of Southern California}, Los Angeles, CA, USA \\
Advisors: \href{http://cacs.usc.edu/people/nakano.php}{Aiichiro Nakano} and \href{http://www.emilio.ferrara.name/}{Emilio Ferrara} \\
Thesis: \href{http://digitallibrary.usc.edu/cdm/compoundobject/collection/p15799coll89/id/367418/rec/71}{Alleviating the Noisy Data Problem using Restricted Boltzmann Machines} \\

{\bf BE in Information Science and Engineering} \hfill {\em 2012 - 2016} \\
{\it M.S. Ramaiah Institute of Technology}, Bengaluru, India \\

\end{rSection}
%%%%%%%%%%%%%%%%%%%%%%%%%%%%%%%%%%%%%%%%%%%%%%%%%%%%%%%%%%%%%%%%%%%%
% Experience
%%%%%%%%%%%%%%%%%%%%%%%%%%%%%%%%%%%%%%%%%%%%%%%%%%%%%%%%%%%%%%%%%%%%
\begin{rSection}{Experience}

\begin{rSubsection}{University of Southern California}{\em 2018 - 2020}{Research Assistant}{Los Angeles, CA}
\item[] Advisor: \href{https://szesuen.usc.edu/}{Sze-Chuan Suen}
\item Researched on techniques to model the effectiveness of {\em Pre-exposure prophylaxis} (PrEP) on HIV/AIDS outcomes in Los Angeles county.
\item Developed interactive web application that allows online modeling of HIV/AIDS outcomes.
\item Designed end-to-end deep learning pipeline to predict mortality of patients at Sutter Health based on patient characteristics, vitals, labs and interventions.
\end{rSubsection}

\begin{rSubsection}{Information Sciences Institute}{\em January - May, 2019}{Directed Research Assistant}{Marina Del Rey, CA}
\item[] Advisors: \href{https://viterbi.usc.edu/directory/faculty/Lucas/Robert}{Robert F Lucas} and \href{http://www-scf.usc.edu/~jeremyjl/}{Jeremy Liu}
\item Modeled large-scale reactive molecular dynamics (RMD) simulations data set of $MoS_{2}$ monolayer to be able to denoise grain boundaries and defects.
\item Used restricted Boltzmann machines (RBM) and limited Boltzmann machines (LBM) which was sampled using D-Wave adiabatic quantum annealer (AQA).
\item Improved the performance of the LBM by finding techniques to efficiently reassign its hidden units to the qubits of AQA.
\end{rSubsection}

\begin{rSubsection}{Indian Statistical Institute Bangalore Center}{\em 2017 - 2018}{Research Assistant}{Bengaluru, India}
\item[] Advisor: \href{https://www.isibang.ac.in/~saroj.meher/index.html}{Saroj Kumar Meher}
\item Conducted exploratory research on techniques to model geological data using fuzzy neural network and other deep learning models.
\end{rSubsection}

\begin{rSubsection}{M.S. Ramaiah Institute of Technology}{\em 2016 - 2018}{Research Associate}{Bengaluru, India}
\item[] Advisor: \href{http://www.msrit.edu/department/faculty-detail.html?dept=ise&ID=8}{Krishnaraj P.M.}
\item Improved the performance of existing methods to identify influentials in a social network using several unsupervised and statistical machine learning approaches.
\item Responsible for successful organization of the materials for a book on social network analysis focusing on the practical applications of several theoretical concepts.
\end{rSubsection}

\end{rSection}
%%%%%%%%%%%%%%%%%%%%%%%%%%%%%%%%%%%%%%%%%%%%%%%%%%%%%%%%%%%%%%%%%%%%
% Publications
%%%%%%%%%%%%%%%%%%%%%%%%%%%%%%%%%%%%%%%%%%%%%%%%%%%%%%%%%%%%%%%%%%%%
% Preprints
%%%%%%%%%%%%%%%%%%%%%%%%%%%%%%%%%%%%%%%%%%%%%%%%%%%%%%%%%%%%%%%%%%%%
%\begin{rSection}{Preprint}



%\end{rSection}
%%%%%%%%%%%%%%%%%%%%%%%%%%%%%%%%%%%%%%%%%%%%%%%%%%%%%%%%%%%%%%%%%%%%
% Journal articles
%%%%%%%%%%%%%%%%%%%%%%%%%%%%%%%%%%%%%%%%%%%%%%%%%%%%%%%%%%%%%%%%%%%%
\begin{rSection}{Journal articles}

\item \textbf{Ankith Mohan}, Aiichiro Nakano, Emilio Ferrara.
``\href{https://arxiv.org/pdf/2011.10549.pdf}{Graph signal recovery using restricted Boltzmann machines}''. 
%\textbf{arXiv Preprint}.
{\it Expert Systems with Applications} (2020) (under review)

\item Jeremy Liu, \textbf{Ankith Mohan}, Rajiv K. Kalia, Aiichiro Nakano, Ken-ichi Nomura, Priya Vashishta, and Ke-Thia Yao. ``\href{http://cacs.usc.edu/papers/Liu-QLBM-CMS20.pdf}{Boltzmann machine modeling of layered MoS2 synthesis on a quantum annealer}''. {\em Computational Materials Science} 173 (2020): 109429.

\item Krishnaraj P. M., \textbf{Ankith Mohan}, and Srinivasa K.G. ``\href{https://www.researchgate.net/publication/318762687_Performance_of_procedures_for_identifying_influentials_in_a_social_network_prediction_of_time_and_memory_usage_as_a_function_of_network_properties}{Performance of procedures for identifying influentials in a social network: prediction of time and memory usage as a function of network properties}''. {\em Social Network Analysis and Mining} 7, no. 1 (2017): 34.

\end{rSection}
%%%%%%%%%%%%%%%%%%%%%%%%%%%%%%%%%%%%%%%%%%%%%%%%%%%%%%%%%%%%%%%%%%%%
% Textbooks
%%%%%%%%%%%%%%%%%%%%%%%%%%%%%%%%%%%%%%%%%%%%%%%%%%%%%%%%%%%%%%%%%%%%
\begin{rSection}{Textbook}

\item Krishnaraj P.M., \textbf{Ankith Mohan}, and Srinivasa K.G. \href{https://link.springer.com/book/10.1007/978-3-319-96746-2}{\em Practical Social Network Analysis with Python}. Springer International Publishing, 2018.

\end{rSection}
%%%%%%%%%%%%%%%%%%%%%%%%%%%%%%%%%%%%%%%%%%%%%%%%%%%%%%%%%%%%%%%%%%%%
% Open-source projects
%%%%%%%%%%%%%%%%%%%%%%%%%%%%%%%%%%%%%%%%%%%%%%%%%%%%%%%%%%%%%%%%%%%%
\begin{rSection}{Open-source projects}

\begin{rSubsection}{denoiseRBM}{\url{https://github.com/ankithmo/denoiseRBM}}{}{}
\item Model-agnostic pipeline to recover graph signals by exploiting content-addressable memory property of RBM and the hidden layer representations of a deep neural network (DNN).
\item Pipeline can be used for any dataset but is particularly effective for graph-structured datasets.
\item Requires the deep neural network to be trained on \textit{clean} data, data which is free of noise.
\end{rSubsection}

\begin{rSubsection}{estimateMI}{\url{https://github.com/ankithmo/estimateMI}}{}{}
\item Implementation of Ziv Goldfeld, Kristjan Greenewald, Yury Polyanskiy. (2019) ``\href{https://arxiv.org/abs/1810.11589}{Estimating Differential Entropy under Gaussian Convolutions}''.
\item Estimating the mutual information between the input layer and each of the hidden layer representations using a {\em noisy} DNN, where additive white Gaussian noise (AWGN) is injected to each of these representations.
\item Extending the work to estimate information flow in graph neural networks.
\end{rSubsection}

\begin{rSubsection}{Deep Pommerman}{\url{https://deep-agents.github.io/}}{}{}
\item Solving the game of \href{https://www.pommerman.com/}{Pommerman} using deep reinforcement learning.
\item Cooperated with five teammates to design both curriculum learning and reward engineering methods to progressively train the game agent. 
\item Trained agents that used imitation learning or Monte Carlo tree search methods to track and eliminate opponent agents.
\end{rSubsection}

\end{rSection}

%%%%%%%%%%%%%%%%%%%%%%%%%%%%%%%%%%%%%%%%%%%%%%%%%%%%%%%%%%%%%%%%%%%%
% Skills
%%%%%%%%%%%%%%%%%%%%%%%%%%%%%%%%%%%%%%%%%%%%%%%%%%%%%%%%%%%%%%%%%%%%
\begin{rSection}{Skills}

\begin{rSubsection}{Languages}{}{}{}
\item[] Python, R, Matlab, C++
\end{rSubsection}

\begin{rSubsection}{Libraries}{}{}{}
\item[] {\it Deep learning}: PyTorch, Tensorflow
\item[] {\it Geometric deep learning}: PyTorch geometric, Deep Graph Library, Graph Nets
\item[] {\it Visualization}: Dash, R Shiny
\end{rSubsection}

\end{rSection}


%\begin{tabular}{ @{} >{\bfseries}l @{\hspace{6ex}} l }
%Computer Languages & Prolog, Haskell, AWK, Erlang, Scheme, ML \\
%Protocols \& APIs & XML, JSON, SOAP, REST \\
%Databases & MySQL, PostgreSQL, Microsoft SQL \\
%Tools & SVN, Vim, Emacs
%\end{tabular}

\end{document}
